\documentclass[letterpaper,10pt,titlepage, draftclsnofoot,onecolumn]{IEEEtran}

\usepackage{graphicx}
\usepackage{amssymb}
\usepackage{amsmath}
\usepackage{amsthm}

\usepackage{alltt}
\usepackage{float}
\usepackage{color}
\usepackage{url}

\usepackage{balance}
\usepackage[TABBOTCAP, tight]{subfigure}
\usepackage{enumitem}
\usepackage{pstricks, pst-node}

\usepackage{geometry}
\geometry{textheight=8.5in, textwidth=6in, margin=.75in}

\usepackage{hyperref}
%credit to http://tex.stackexchange.com/questions/48152/dynamic-signature-date-line
\newcommand*{\SignatureAndDate}[1]{%
    \par\noindent\makebox[2.5in]{\hrulefill} \hfill\makebox[2.0in]{\hrulefill}%
    \par\noindent\makebox[2.5in][l]{#1}      \hfill\makebox[2.0in][l]{Date}%
}%
\begin{document}

\title{Requirements Document}
\author{Kyle Nichols, Shannon Ernst, Javier Franco}
\date{October 28, 2016}
\maketitle
\section{Introduction}
\subsection{Purpose}
This software requirements specification is intended to define the requirements of the project of developing a database 
of surveys for STEM Academy. Defined requirements will allow for a contract between us, the developers, and Stem Academy, 
our clients, on what STEM Academy wants us to deliver in their desired software. This document is intended for review and
reference by both the developers and the clients.
\subsection{Scope}
The product outlined in this requirements document will be [OUR NAME HERE]. The product will need to be able to store survey
questions and responses from students to those questions. It will need to take those responses and generate data reports
based on queries to the database, such as summary or demographic data. This product is intended to ease the distribution of
surveys and collection of data for STEM Academy so they can more easily provide this information to sponsors and on
applications for funding. % I'm pretty sure there's a better way of explaining this. -Kyle. 

The software products that will be produced include a survey generator, a database that stores surveys questions and responses from students to those questions, and a report generator that generates results based on queries to the database.  
The results of the report generator will be presented as raw data or a summary of the data (average, count, sums) and it will not produce any kinds of graphs which include pie charts, bar graphs, and line graphs.  
The goal of the software is to help STEM Academy to obtain more funding for their camps/clubs/programs by helping ease the distribution of surveys and collection of data, so that they can present this information to their sponsors and on their applications for obtaining more funding. 
The benefits that the software will have on STEM Academy is that they will be able to produce reports quickly which will save them a significant amount of time by not having to tabulate results by hand and it will prevent survey results from being untabulated.
 In addition, they will no longer have to verify their survey results since the software will generate results that are 100\% accurate.
 Another benefit that the software will have on STEM Academy is that student survey responses and survey questions will be stored in a secure database instead of a file system and their personal information will be encrypted to maintain it confidential from outside users. 
% Would this work for the scope - Javier 

\subsection{Definitions}
\begin{itemize}
\item Contract - A legally binding document that is agreed upon by the customer and developers.  %Added the terms contract, customer, and developers -Javier 
\item Customer - The person, or persons, who will be receiving the final product and provide feedback to the developers, so that they can meet the requirements for their desired product. 
\item Data analyst - A user that uses the product to generate a report based on survey response data.
\item Database - A collection of information that is organized, so that it can be easy to modify and find information. 
\item Developers - A group of people who will build a product for a customer.
\item Matrix question - A multi-part question where each part is a row that has several options for response. All parts have the
same answer choices.
\item Multiple choice question - A question where the response is a selection from different options, and only one selection
is possible.
\item Query - Extracts information from the database based on the criteria that the user entered.
\item Question - A grammatical question that is represented in our database with associated responses.
\item Question type - Questions can vary based on the kind of response desired (multiple choice, text entry, etc.). % Shannon, is this what you meant? -Kyle.
\item Survey builder - A user that creates surveys on the database.
\item Survey taker - A user that provides answers to surveys existing on the database. These responses are stored in
the database.
\item Template question - A question that follows a general structure. This can be edited to change some wording that can
vary between surveys (such as the material taught) while the general structure stays the same.
\item Text entry question - A question where the response is a box that allows the user to answer with a string of text.
\item User - A person that will be interacting and using the product.
\end{itemize}
\subsection{References}
% Do we have any outside information in this that must be cited? I haven't seen any. -Kyle.
\subsection{Overview}
The software requirements specification is organized as follows with a description of the overall product, the specific requirements of the
product, and functional requirements of the product. The product description will include % finish after rest of the document done -Kyle.

The remaining parts of the software requirements specification document include the overall description of the product followed by the specific requirements of the product.
 The purpose of the overall description section is to provide a background for the requirements that will be listed in the specific requirements section and also to explain any factors that can influence the product and its requirements. 
 In the overall description section, it contains the product perspective, products functions, user characteristics, constraints that will limit the development of the product, and any assumptions and dependencies that we have that if changed will require altering our requirements. 
The specific requirements section contains external user interface requirements, functional requirements, design constraints, software system attributes, and other requirements. 
The purpose of the specific requirements section is to help us with our design process by satisfying the requirements we listed and to test whether or not we were able to meet our software requirements.  %Not sure if whether or not we should give a detailed description of the remaining SRS -Javier 
\section{Overall Description}

\subsection{Product Perspective}
The product will be wholly self-contained, as it is intended as a product to specially satisfy the needs of STEM Academy's survey
process. The user interface should be designed in such a way that user's do not need a knowledge of SQL queries to be able
to query the database to generate reports. An example would be a list of common actions that they can perform on the database
and text fields to retrieve information necessary for the SQL query. This web user interface will connect with the database management
software we use for our product.
\subsection{Product Functions}
The basic functionality of the product will be that a user can log on to a web page as either a survey builder,
a survey taker or a data analyst. The survey builder can then create surveys that can be stored on the database as a survey with a
list of questions associated with it. A survey taker can log on to the web site and take a survey stored on the database,
while the responses they provide to survey questions are also stored on the database with an association with the user
and the question involved. A data analyst can log on to the web site and query the database to generate a report on
information desired.
\subsection{User Characteristics}
The users of the software follow two basic characteristic sets. The first are survey builders and data analysts that are members of
STEM Academy. These users are educated and have experience in creating surveys and understanding data from surveys,
however they are also non-technical in regards to this kind of product. The other type of user are the survey takers, who are
generally either grade school students or the parents of the students. The education and the experience in STEM fields
can vary in this type of user, from minimal to deep, though most will be minimal. The technical knowledge will also vary, though
most are likelly to have the required knowledge to navigate a web page.
\subsection{Constraints}
% This one I am not sure of. The only thing I can think of is the possibility of regulatory policies if we get to use on-campus resources -Kyle.
\subsection{Assumptions and Dependencies}
For this product, we assume that we will have access to a resource on campus that can store a large amount of information on
surveys and responses. % This is all I've got -Kyle.
\section{Specific Requirements}
% I don't get what makes this different from other sections. Might ask Kristin about this one as well. -Kyle.
\subsection{External Interface Requirements}
% Same as above. I feel like I could use a better explanation than the standards doc here. -Kyle.
\subsubsection{User Interface}
The user interface will be a website. There are two views: administration and survey taker. 
The Administration view is for those creating surveys and analyzing data. They will input a 
username and password to gain access. They will then be navigating the site via mouse clicks 
and keyboard interaction. There will not be support for a touch screen interface. The survey
taker view is only for surveys. The survey will be distributed through a website link which 
will allow temporary access to that particular survey for the survey taker to take. The 
interaction will be via mouse clicks and keyboard. There will not be support for a touch screen
interface.

\subsection{Functional Requirements}

\subsubsection{Survey Generation}
%\subsubsection{Overall}
Surveys should be able to be made from scratch and saved. Previously created surveys should be editable 
meaning questions should be able to be added/deleted, moved in ordering and the text of the question 
and answer options should be able to change. These surveys should also be able to be saved as well. 
Surveys may be deleted once created. A survey should be printable at anytime as well as publishable to a
web link which can be distributed. All surveys and questions will be saved to a database.
%\subsubsection{Structure and Formatting}
The user should be able to add questions to and delete questions from a survey. The questions should be able to
be reordered. The user should be able to add display logic to any question in the survey. Display logic means the
question will appear based on the answer of previous questions.
%\subsusection{Question}
There are three types of questions: matrix, multiple choice and text entry. Each question is comprised of the 
question text and the answer options text. The question text has the option of having a changeable subject
field where the everything about the question stay exactly the same except for the subject. Questions should
be savable and editable. \\

\subsubsection{Data Collection, Analysis and Report Generation}
A survey taker should be able to take a survey and the results stored to a database. The data should be queriable
via a user interface. Valid queries and subqueries include viewing data by: camp/club/program, student, demographic 
information, and questions. The results of these queries should be savable to a report which will allow titling and captioning 
of the data. The queries may result in raw data or summary data (averages, sums, counts, etc.) per the users specification 
in the user interface. These reports should be savable, editable and printable at all times. 


\subsection{Performance Requirements}
The web page should be up at least 99\% of the time. 95\% of the database queries should take less than 2 seconds 
to generate a report. The sytems should store 100\% of the responses correctly and queries to the database should produce results that are 100\% accurate. %added that the queries should be 100% accurate  -javier 

\subsection{Design Constraints}
The database will be constrained by the storage limits of the database system utilized.

\subsection{Software System Attributes}
 The system should be available to access 95\% of the time through the web page. For security purposes, 
students' personal information and users passwords should be encrypted on the data base side. %encrypt user passwords -Javier 
The software should require ease of maintenance for a non-technical background so the clients have the ability to resolve any problems after the completion of the project.

\subsection{Other Requirements}
\newpage
\section{Signatures}
By signing below, I agree to the ideas and concepts presented in this document and the list of deliverables. \\
\SignatureAndDate{Client Signature}
\vspace{.2in}
\SignatureAndDate{Team Member 1 Signature}
\vspace{.2in}
\SignatureAndDate{Team Member 2 Signature}
\vspace{.2in}
\SignatureAndDate{Team Member 3 Signature}


\end{document}
