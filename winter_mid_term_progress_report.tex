\documentclass[letterpaper,10pt,serif, draftclsnofoot,onecolumn, compsoc, titlepage]{IEEEtran}

\usepackage{graphicx}
\usepackage{amssymb}
\usepackage{amsmath}
\usepackage{amsthm}

\usepackage{alltt}
\usepackage{float}
\usepackage{color}
\usepackage{url}

\usepackage{balance}
\usepackage[TABBOTCAP, tight]{subfigure}
\usepackage{enumitem}
\usepackage{pstricks, pst-node}

\usepackage{geometry}
\geometry{margin=.75in}

\usepackage{hyperref}
\usepackage{verbatim}

\title{The STEM Academy Data Solution}
\author{Progress Report \\ Shannon Ernst, Kyle Nichols, Javier Franco\\ 7 December 2016\\ CS 461 Fall 2016\\ Group 48}

\begin{document}
\maketitle
\begin{abstract}
It has been a term of foundation building for the STEM Academy Data Solution. The project was assigned, requirements were outlines and a design has been produced. Though much work is left to do, this document outlines the current progress.
\end{abstract}
\newpage
\section{Purpose and Goals}
The STEM Academy program in Corvallis, Oregon, seeks to provide education in science, technology, engineering and math to the K-12 community through various programs, campus, and clubs.
They are self funded, relying on grants, donations and other sources, many of which require results-based data on the success of STEM Academy programs.
This data is collected from surveys that participants take while they are enrolled in one of STEM Academy's camps and other programs.
Currently, these surveys are administered on paper and the responses are tabulated by hand by volunteers.
This has not been an effective way of collecting current data to help secure funding and many of the surveys go untabulated.
The STEM Academy does have the ability to register participants online through Ideal Logic, but this is this does not help them with tabulating survey data.
They need a system which will allow them to electronically create surveys, a database to house all of the data from those surveys, and a report generation system so they can view the most current data in a timely and meaningful way.

A website which will allow the STEM Academy to create and distribute surveys, as well as view helpful data results, is the best solution.
The website should allow them to create questions that can be included across multiple surveys.
The questions should be able to be templated so that only one part of the question need change according to the topic of the survey.
Surveys should be able to be distributed via a web link or printed.
After a survey is taken, the data should be stored in a database according to the type of program.
This data should be queriable based on custom queries.
The user should be able to make a query based on the question and collected demographics as well as filter by program.
This service should be user friendly; the user should not need to know any SQL or other querying language in order to get data from the database.
After a query, the data should be able to be saved and added to a printable report so that it can actually be used in grant proposals and marketing material.
This website will be an all in one package for collecting and reporting on data in a timely fashion which is what the STEM Academy truly needs.
By the Engineering Expo, we will be able to present this website and show the full path of data collection, from creating the survey to querying to report generation.

\begin{comment}
\section{Weekly Summary}
\subsection{Week 1}
Over winter break Kyle set up a prototype of the database on his ONID account since we have not heard back from Central Services, so that our group would not lose momentum waiting on them.
He also studied AngularJS for the first week in order to be more useful to the group.
This week Shannon contacted Central Services since we had not heard back from them and the reason was that they had lost our ticket.
They told her that we can work on a LAMP stack and extend the pre-existing domain for the STEM Academy.
This would mean that the code would be done using php and mysql, but there is nothing inherently wrong with this
 For this week Javier worked on an alternative design for querying information since he thinks his previous design will not work.
The new design for a query consists of three drop downs for selecting a question, response, and an operator for a query which is similar to the one that Qualtrics uses.
He also studied Qualtrics to see what operators the tool uses for querying information and the flow on generating a result for a query.
Our group established a weekly meeting with our client for Tuesdays at 9:00 am.
The next step that our group will be taking is creating the website through Central Services. 

\subsection{Week 2}
This week Kyle created the initial setup of our web page and handed it off to Shannon and Javier to create their respective pages, starting with some basic templates and functionality.
The current state consists of mostly links going to empty pages.
Except for the survey page which is able to add text to the survey text, but it cannot save anything and do more than one question.
Central Services contacted Shannon and told her that we can do the LAMP stack which is what our group is going to go with.
This week Javier wacthed another Angular 2 video and also started working on the query form page, but encountered a problem while trying to query the database for the camps to list on a drop down menu.
This week our group met up with our client for our first weekly meeting and we updated them on our plans moving forward and we also showed them the progress we made on the web pages.
Our groups goals for the upcoming week is to create a better HTML frame work, so that we can illustrate the flow of the program to the client. After this we will start working on functionality. 

\subsection{Week 3}
For week 3 Javier added some functionality to the report generation page which now allows a user to delete or add a query.
He also created the functionality that allows admins to login or logout of their accounts.
He plans on continuing adding more functionality to the report generation page for the upcoming week.
Some issues that he faced this week was that by not having the correct property value for the files it resulted in the files not functioning which took him a while to figure out.
This week Kyle updated the database to include a Camp table based on discussing the organization of the database and their needs with the client
 He also made other modifications to other tablese based on this change or demographic information that was missing.
For this weekend and early next week Kyle plans on getting code written to take JSON objects and convert that into a SQL query, as well as creating the code to take a SELECT query result and turn it into a useful (probably JSON) object.
This will help provide the basic functionality for interaction with the database from the rest of the service.
This week Shannon started implementing the functionality for adding questions to the survey.
Over the weekend she is going to continue working on this as well as implementing the preview survey.
In addition, she plans on calling Ideal Logic to figure out how to export the enrollments of studnets. 

\subsection{Week 4}
This week Kyle wrote the basic code for adding to the database and selecting from the database
 However, this is using hard-coded table entries due to Shannon working on the creation of the JSON object that he plans to parse from.
This week Javier fixed the add a camp page and an issue he encountered was that it was difficult debugging the problem that the page had.
The meeting with he client helped give Javier a better understanding of the queries that the client wants to be able to do, but is worried about coming up with a good design that will help meet their needs
For the upcoming week Javier plans on focusing more on report generation since this is the part of the project he is responsible for.
For this week Shannnon implemented the beginnings of the survey creation which is now able to generate questions dynamically and also create surveys.
In addition, she was able to construct the JSON object.
For the survey generation she still needs to implement saving a survey, edit pre-existing surveys, and have the ability to recall surveys.
She stills needs to contact Ideal Logic.
This week we met with our client to provide updates and to demo what we have.
Currently everything is going well for our capstone project. 

\subsection{Week 5}
This week our group met up with our client and they provided feedback on the report generation page design and during the meeting Shannon drew an alternative query structure based on their feedback.
Javier will have to make some changes to the report generation page, so that the client is able to do the queries that they requested.
Our client also emailed us a follow up about the things they want to be able to query along with examples.
This week Shannon contacted Central Services to get a sub domain.
Both Shannon and Kyle worked on getting the PHP working for posting a JSON object to the database table.
However, they ran into a problem where the POST data was coming back empty in the PHP code and were unable to find the issue causing it.
The plans that our group has for the upcoming week is to make revisions to our documents, create the progress report document, record the demo video, work on the capstone project, and to solve the issue with the POST data of a JSON object. 

\subsection{Week 6} 
\end{comment}

\section{Weekly Progress Summary}
\subsection{Shannon Ernst}
\subsubsection{Week 1}
This week I contacted Central Services as we had not heard from them.
Our ticket had been lost.
They told us that we can work on a LAMP stack and extend the preexisting domain for the STEM Academy.
This means that the code would be done in PHP and MySQL.
There is nothing inherently wrong with this.
We also established a weekly meeting with our client for Tuesday at 9 am.
The next steps are creating the website through Central Services.
\subsubsection{Week 2}
This week we met with our client to show a basic mock of what we are doing for the web app.
We are going to continue working primarily in HTML for the moment.
Central Services says we can do the LAMP stack which is what we are going to go with.
Goals for the coming week is to have a better html frame work so we can illustrate the flow of the program to the client.
We will then work on functionality.
\subsubsection{Week 3}
This week I started implementing functionality for adding questions to the survey.
Over the weekend I'm going to continue working on this as well as implementing preview survey.
I need to call Ideal Logic to figure out how to export enrollments.
We meet with our client again next week.
\subsubsection{Week 4}
This week I implemented the beginnings of survey creation.
I have constructed the json and can generate questions dynamically and create surveys.
I still need to be able to recall surveys, save and edit preexisting ones.
I still need to call Ideal Logic. We met with our client this week to demo and provide updates.
All is going well.
\subsubsection{Week 5}
This week we contacted central services to get the sub domain.
We are working on getting php working for posting json.
We sketched a query structure with the client.
This coming week we will do the progress report and wrap up alpha with successful PHP POSTs.

\subsection{Javier Franco}
\subsubsection{Week 1}
This week I looked at creating an alternative design for creating the search form since I thought that my previous design would not work.
The new design is based on selecting a question to query first for a survey.
Then selecting an operator for the query and specifying a value that is going to be queried through a drop down menu or by manually entering a value.
I compared this design idea to Qualtrics and noticed that it uses this set up.
I also studied the operators that Qualtrics uses for the different types of questions that a user will be able to create and how it carries out multiple field queries.
The plan for next week is to start developing our webpages, so that we can have something to show to the client and I will also be studying more on how to use Angular JS 2 to get a better understanding.
\subsubsection{Week 2}
My plan for next week is to create a weekly schedule and to set aside 4-6 hours a day to work on the project to better manage my time and to make progress on the project.
This week I watched another tutorial video on Angular 2 and I worked on creating query form page, but encountered errors while trying to query the clubs, camps, and programs for the drop down list.
I created three tables for the club, program, and camp that I need to include in Kyle's database.
I felt that I was not as productive as I should have been and I will make it up by working on the project over the weekend.
\subsubsection{Week 3}
This week I added some functionality to the report form generation page.
The page now allows a user to delete/add a query.
I also created the functionality that allows administrative users to login/logout.
Some issues that I faced this week was that by not having the appropriate properties on the files and folders it effected the results produced by the .js and .php files which took me a while to figure out.
My plan for the weekend is to add more functionality to the page by allowing a user to select a camp and a survey to query.
\subsubsection{Week 4}
This week I was able to get the adding a camp page to work and an issue that I encountered was that it was time consuming finding the error that caused it because webpages are hard to debug.
Our group met up with the clients and I now have a better understanding of the things they want to be able to query.
However, I am worried that it will be difficult to come up with a good design that satisfied their demands for the report generation webpage.
For the upcoming week I will be focusing more on the part of the project that I am responsible for and my goals for the week is to complete the query functionality for a survey.
\subsubsection{Week 5}
This week our group met up with the client and they gave me feedback on my report generation design.
I am going to have to change my design in order to meet their requests of the stuff they want to be able to query.
The plans for the upcoming week are to create the demo video, make revisions on our documents, make a progress report, and also work on the capstone project.
For the upcoming week I am worried that I will not have time to work on the capstone project since we have to make revisions to our documents, make a progress report, and a demo video.
I will also be having a midterm for Linear Algebra that I need to study for.

\subsection{Kyle Nichols}
\subsubsection{Week 1}
The first thing I did for the project was over winter break by setting up a prototype database on ONID.
I did this through creating the tables as outlined in the design document: Survey, Question, Contains, Responder, Student, Parent, Response, Multiple\_Choice, Text, and Matrix.
These tables were created empty.
I did not yet add any example data to the database.
The ultimate plan was to store the database somewhere provided by Central Services, but since we still had not heard back from them, I still needed something so as not to lose forward momentum.
\subsubsection{Week 2}
The next week, I created the initial setup of our web page and handed it off to the others to create their pages, starting with some basic templates and functionality.
This template-like code came from what I had created for another class at my Engineering public\_html/ space (located at web.engr.oregonstate.edu/~nichokyl).
This allowed us to have a place to create some prototype code and my team mates followed with their own public\_html/ pages.
We still stored all of our web service code in public\_html/ in our github directory, but used our own Engineering web pages to test our code for each of our purposes.
At this point, most links went to blank pages, though the page for adding questions and surveys did allow for filling in survey text, but could not save anything or create more than one question.
\subsubsection{Week 3}
During this week I updated the database to include a Camp table based on discussing the needs of the database with the client.
In particular, they needed a way to organize surveys as based on camp.
My original design of the database only accounted for Surveys in a general sense and not their use in camps more specifically.
This also required that I create an S\_Use table to account for the many-to-many relationship that Surveys are used in Camps and Camps use surveys.
Some other modifications were also made to tables based on this change, such as adding and removing attributes.
These include removing a 
Plus, I added more demographic information to some tables based on an exampe survey the client provided.
\subsubsection{Week 4}
This week I wrote the basic code for adding to the database and selecting from a database.
I started this component using hard-coded table entries while Shannon was setting up the JSON creation that I would parse from.
\subsubsection{Week 5}
On Sunday of week five, I worked with my group to make progress on creating interaction between survey generation and adding those surveys to a database.
This included creating the code that would parse from the JSON objects provided by survey generation to insert to the database.
The SQL query code seemed to work on adding to tables in the database, but we ran into a problem where POST data PHP wanted to use was empty, so nothing could actually be parsed from it.
\subsubsection{Week 6}
Unfortunately, we currently have not resolved the issue wtih POSTing the JSON objects to the PHP file that adds to the database.
However, in the mean time, I have a temporary fix for this issue:
If a .json object exists in the same directory as the source code inserting into the database, I can get the contents of that file and store that in a JSON object to be parsed.
This is not how we will want to implement the JSON transfer in the final version, but for now I can use this to get insertion into the database functional while we continue to test our POST issue.

\section{Left to Do}

\subsection{Shannon Ernst}
\subsection{Javier Franco}
\subsection{Kyle Nichols}
The database organization is good and there are no current plans to change it considerably.
Some changes may continue to occur as we continue to develop our service and test it.
Several issues have occurred already where modifications needed to be made, and more may still happen.
Howver, looking forward and the work left to do on database interaction, the connection between the survey generation and report generation components and the database needs to be implemented fully.
Currently, the code for generating SQL queries is written, but it is basic and does not take requests from other pages for queries.
The code will needed to be expanded, taking in JSON objects and parsing those objects to know what kind of query or queries are being requested (i.e. a \emph{SELECT} or \emph{INSERT} and what information to send in those queries.

\section{Retrospective}
\begin{center}
    \begin{tabular}{ | p{5cm} | p{5cm} | p{5cm} |}
    \hline
     Positive & Delta & Action \\ \hline
  	Every assignment was completed & Our writing is not always concise and does not always accurately convey our ideas & To solve the writing issue we will consult more with the instructors and TAs as well as go to the writing lab \\\hline
	We have a solid understanding of our project & We had a really tough time with the IEEE formats & To solve the IEEE formatting issue, we will ask more questions in the future and take notes on the answer \\ \hline
	We have a good relationship with our client & We could still make sure to more frequently update our clients on our progress & We will make sure to have email or in-person communication once a week \\ \hline
	&We need to do a better job of scheduling and keeping to the schedule& To fix the scheduling we will start having a calendar on github which we can all monitor and stick to \\ \hline
    \end{tabular}
\end{center}

\section{Current Status}


\end{document}
