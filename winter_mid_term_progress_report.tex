\documentclass[letterpaper,10pt,serif, draftclsnofoot,onecolumn, compsoc, titlepage]{IEEEtran}

\usepackage{graphicx}
\usepackage{amssymb}
\usepackage{amsmath}
\usepackage{amsthm}

\usepackage{alltt}
\usepackage{float}
\usepackage{color}
\usepackage{url}

\usepackage{balance}
\usepackage[TABBOTCAP, tight]{subfigure}
\usepackage{enumitem}
\usepackage{pstricks, pst-node}

\usepackage{geometry}
\geometry{margin=.75in}

\usepackage{hyperref}
\title{The STEM Academy Data Solution}
\author{Progress Report \\ Shannon Ernst, Kyle Nichols, Javier Franco\\ 7 December 2016\\ CS 461 Fall 2016\\ Group 48}

\begin{document}
\maketitle
\begin{abstract}
It has been a term of foundation building for the STEM Academy Data Solution. The project was assigned, requirements were outlines and a design has been produced. Though much work is left to do, this document outlines the current progress.
\end{abstract}
\newpage
\section{Purpose and Goals}
The STEM Academy in Corvallis, Oregon seeks to provide education in science, technology, engineering and math to the K-12 community through various programs and camps/clubs. It is self funded, relying on grants, donations and other sources, many of which require results based data on the STEM Academy. This data is collected from surveys that participants take while they are taking part in the program. Currently, these surveys are administered on paper and the responses are tabulated by hand by volunteers. This has not been an effective way of collecting current data to help secure funding and many of the surveys go untabulated. The STEM Academy has the ability to register participants online through Ideal Logic but this is the only digital component they have for collecting data. They need a system which will allow them to electronically distribute and print surveys, a database to house all of the data from those surveys, and a report generation system so they can view the most current data in a timely and meaningful way.\\
A website which will allow the STEM Academy to create and distribute surveys as well as view the data is the best solution. The website should allow them to create questions that can be included across multiple surveys. The questions should be able to be templated so that only one part of the question need change according to the topic of the survey. Surveys should be able to be distributed via a web link or printed. After a survey is taken, the data should be stored in a database according to the type of program. This data should be queriable based on custom queries. The user should be able to make a query based on the question and collected demographics as well as filter by programs and camps/clubs. This service should be user friendly; the user should not need to know any SQL or other querying language in order to get data from the database. After a query, the data should be able to be saved and added to a printable report so that it can actually be used in grant proposals and marketing material. This website will be an all in one package for collecting and reporting on data in a timely fashion which is what the STEM Academy truly needs. By the Engineering Expo, we will be able to present this website and show the full path of data collection, from creating the survey to querying to report generation.

\section{Weekly Summary}
\subsection{Week 1}


\subsection{Week 2}


\subsection{Week 3}

\subsection{Week 4}


\subsection{Week 5}


\subsection{Week 6} 


\section{Retrospective}
\begin{center}
    \begin{tabular}{ | p{5cm} | p{5cm} | p{5cm} |}
    \hline
     Positive & Delta & Action \\ \hline
  	Every assignment was completed & Our writing is not always concise and does not always accurately convey our ideas & To solve the writing issue we will consult more with the instructors and TAs as well as go to the writing lab \\\hline
	We have a solid understanding of our project & We had a really tough time with the IEEE formats & To solve the IEEE formatting issue, we will ask more questions in the future and take notes on the answer \\ \hline
	We have a good relationship with our client & We could still make sure to more frequently update our clients on our progress & We will make sure to have email or in-person communication once a week \\ \hline
	&We need to do a better job of scheduling and keeping to the schedule& To fix the scheduling we will start having a calendar on github which we can all monitor and stick to \\ \hline
    \end{tabular}
\end{center}

\section{Current Status}


\end{document}