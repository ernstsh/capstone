\documentclass[letterpaper,10pt,serif, draftclsnofoot,onecolumn, compsoc, titlepage]{IEEEtran}

\usepackage{graphicx}
\usepackage{amssymb}
\usepackage{amsmath}
\usepackage{amsthm}

\usepackage{alltt}
\usepackage{float}
\usepackage{color}
\usepackage{url}

\usepackage{balance}
\usepackage[TABBOTCAP, tight]{subfigure}
\usepackage{enumitem}
%\usepackage{pstricks, pst-node}

\usepackage{geometry}
\geometry{margin=.75in}

\usepackage{hyperref}
\usepackage{tikz}
\usetikzlibrary{shapes, positioning, calc}
\colorlet{lightgray}{gray!20}
\setcounter{page}{87}
\title{The STEM Academy Data Solution}
\author{User Guide \\ Shannon Ernst, Kyle Nichols, Javier Franco\\ 2 June 2017}

\begin{document}

\maketitle

\newpage
\tableofcontents
\newpage

\section{Introduction}
The following is a user guide for use of the STEM Academy Data Solution.
The STEM Academy Data Solution is accessible through the web site datasolutions.stemacademy.oregonstate.edu
As a web application, there is no need for installation: the site will be fully accessible from Google Chrome.
There are two primary views to the web service: an admin view and a student view.
Each of these will be described in its own section, with the pages contained in each view further described in their own subsections.
The admin view is accessible by clicking the "Admin" button on the front page, while the student view is
 accessible by selecting the "Student" button on the front page.

\section{Admin View}
After clicking the button to enter the admin view, the first page seen is a login page.
Admins registered on the web service will be given a username and password for logging in.
Entering these and clicking "submit" will bring an admin to the main admin page if their username
 and password combination is successfully authorized. Also on the login page are the "Reset" button,
 which clears the input fields, and an "Exit" button that returns the user to the front page.
\subsection{Main Admin Page}
The main page displays a list of options that an admin can choose from which includes adding a new admin, deleting an existing admin, adding a new camp, edit/create a survey, edit/create a report, and logging out.
If an admin clicks on one of the options they will be redirected to its corresponding webpage. 
\subsection{Add Admin}
The only admin accounts that are allowed to create new admins are Catherine Law and Carole Rodriguez.
To create a new admin enter their first name, last name, user name, and password into the corresponding input fields.
Then click the "Submit" button. If Carole or Catherine admin accounts clicked the "Submit" button they will received a pop up message that will say that they have succesfully created a new admin.
Any other admin account that clicks the "Submit" button will receive an error message that tells them that they are not allowed to create new admins.
To return to the Main Admin Page click the "Exit" button. 
\subsection{Delete Admin}
The only admin accounts that are allowed to delete admins are Catherine Law and Carole Rodriguez any other admins that try to delete an admin will have no effect.
Both of their accounts cannot be deleted by anybody including themselves.
This will prevent deleting all the admin accounts.
If all the admin accounts were to be deleted it will render the web application useless since there are no admin accounts that can log in.
To delete an admin enter their first name, last name, and user name into the corresponding input fields.
Then click the "Submit" button.
To return to the Main Admin Page click the "Exit" button. 
\subsection{Add Camp}
To add a new camp or update a camp click the the "Add Camp" link.
To create a new camp first enter the name of the camp and the date range.
Next upload the .csv file containing the enrollment of the camp  by clicking the "Browse..." button.
Select the file you want to upload and click the "Open" button and click the "Submit" button.
To update a camp select the name of the camp from the drop down menu.
Next enter the new name of the camp into the input field and enter the new date range.
Click the "Browse..." button to upload a new enrollement for the camp. Select the file you would like to upload and click the "Open" button.
To submit your changes click the "Update" button.
To exit the Add Camp page click the "Exit" button. 

Important Note:
After updating a camp roster, be sure to check the list of students for the camp by viewing the drop down menu in the student view (described in section 3).
If there is a student that appears to be missing, there may have been an issue with loading them into the system.
Refer to section 4.2 for troubleshooting solutions.
\subsubsection{Last Minute Changes in Enrollment}
If there is a last minute change in enrollment before a camp, this is fine.
Simply re-upload the .CSV file with the changes in enrollment and click the "Update" button.
This will update the database to reflect the new enrollment.
\subsection{Edit/create a Survey}
By clicking the "Edit/Create a Survey" link on the Main Admin Page, user can create a survey or load a preexisting one.
To load a preexisting survey, select a survey from the "Load Preexisting Survey" drop down at the top of the page.
This will populate all available fields with the information previously provided for that survey.
Any of the data fields may be edited as normal. To create a survey from scratch, fill out all appropriate fields including Title, Camp (from the provided drop down) and whether it is a Pre or Post survey.
From there select a type of question to add from the "Add New Question" drop down. The options are Text, Multiple Choice and Matrix.
Upon selection, a question will  be presented.
Text questions will only provide a single text box to allow for the question text.
Multiple choice will allow for question text and a default of one answer.
To add more answers, click "Add Answer."
For Matrix, select one of the Likert Scales for agreement and provide a description for the matrix.
Questions can be added by clicking, "Add Question."
All questions can be deleted by clicking the "X" button.
Questions can be marked for frequent use by checking the "Frequently Used" box.
A frequently used question can also be added by selecting it from the "Frequently Used Drop Down" found at the bottom of the page.
To preview the survey, click the "Preview" button at the bottom of the page. 
A seperate window will pop up with the survey.
To save the survey click "Save." This will not redirect you from the page.
To exit the survey click the "Exit" button. To pring the survey, click the "Print" button.
This will take you to a standard print dialog box. 
\subsubsection{Demographic Questions}
The STEM Academy Data Solution allows for storing demographic information on students in camps based on questions asked in a survey.
Storing demographics is linked to responses to specific questions, so to store information for a student, use the drop down menu of questions on survey creation to load a question into the survey.
Each of these questions is marked by the demographic information that it saves (i.e. "Gender" or "Race").
The specific questions are:
\begin{itemize}
	\item "Gender"
	\item "Race: Select one or more"
	\item "Ethnicity: (You may check more than one, as appropriate)"
	\item "Free or Reduced Lunch: I am eligible for the Federal Free or Reduced Price Lunch Program?"
	\item "Guardian Education Level: What is the highest level of education that any of your parents or guardians achieved?"
\end{itemize}
If there is a need to modify the question, that is allowed: simply change any wording in the question or its answers and those will be reflected in the survey.
In addition, adding this newly worded question as a frequently used question will preserve the question's identifier in the database and it can still be used as a demographic question.
\subsection{Edit/create a Report}
The Edit/create a Report option redirects the admin to a new page that gives the admin two options to choose from.
The first option is to create a new report by clicking the "Submit" button.
The second option is to edit a report by first selecting the name of a report from a drop down menu and then clicking the "Submit" button below the drop down menu.
This will display the name and query results of the report. \\
1. To add new query results select the name of the camp from the drop down menu. \\
2. Next select the name of the survey from the drop down menu.
If the admin created a pre and post survey for the camp they selected there should be three options shown.
If they did not the only option displayed will be "Both", but this options requires a pre and post survey to exist in order to be used.
The option "Both" is used solely for finding the the change in reponse for matrix questions from a pre-survey to a post-survey.\\ 
3. Select the query type that you would like use by clicking on its corresponding radio button. 
The two types of queries include one for multiple choice and the second is used for matrix and text questions.
Then click the "Add Query" button this will create the query template based on your selection below the text that says "Queries will appear hear:".
To add another query follow the same process.\\
4. If a query template for a multiple choice question was created first select the multiple choice question that you want a query result for from the 1st drop down menu.
Then select one of the corresponding responses from the 2nd drop down menu.
If a query template for a multiple choice and matrix question was created select the question you want a query result for from the drop down menu.
If you no longer need a query template you can delete it by clicking on its corresponding delete button indicated by "-" symbol.
You can also delete all of the query templates currently displayed on the webpage by clicking the "Reset" button.\\ 
5. Select the result type that you would like to be returned for a matrix and multiple choice question.
This includes a count and percentage by clicking on their corresponding radio button.\\
6. Select the demographic information from the drop down menus to filter your returned query result.
This includes gender, parent's highest education, student race, student ethnicity, and free or reduced lunch.\\ 
7. Once everything has been setup click the "Submit" button.
This will return a query result for each query template and they will be displayed below the text that says "Query Results".
The query result for a multiple choice question will return two input boxes.
The first input box contains the multiple choice question that was selected and next to it is a delete button for removing the query result indicated by a "-" symbol.
The result for a matrix and text question will return a table along with a delete button indicated by a "-" symbol.
The table for a matrix question returns a breakdown of the results for each sub question and the table for a text question returns the name of each student along with their response.\\
8. To add a title to the report enter the name of the report in the first input text box that is shown below the text "Query Results".\\ 
9. When an admin has finished generating the query results for their report they can print their results by clicking the "Print" button.
They can also save their report by clicking the "Save" button.
Furthermore, they can return to the Main Admin Page by clicking the "Print" button.\\ 
\subsection{Log Out}
When an admin clicks the Log Out option they will be redirected to the login page. 
\section{Student View}
After clicking the button to enter the student view, a student will be brought to a page with two drop down menus.
The first menu allows them to select their camp.
Only camps currently in session are listed.
After selecting their camp, the second drop down menu will list all of the students enrolled in the camp.
Students can click "Submit" after selecting their name to load the survey.
\subsection{Taking Surveys}
A survey will generated based on date and camp/student selection. Once an appropriate camp and student combination 
has been submitted, the user will be presented with a pre survey if the date is before the last day of the camp. If 
the it is the last day of the camp, the user will get a post survey. All questions will be presented with the 
appropriate entry space. After the survey has been completed the user can click submit and they will be redirected.

\section{Troubleshooting}
The following information is provided in case STEM Academy encounters a bug in using the STEM Academy Data Solution.
Unfortunately, software cannot be bug-free, but we hope the information provided in this section will help users recover from and resolve bugs that may impair use of the software.
\subsection{Browser Support}
The only browser that the system is tested on is Google Chrome, and therefore can only be guaranteed on Google Chrome.
There is a known issue with Firefox where buttons do not work on the home page.
\subsection{Missing Students}
If a student did not get added to a camp and is missing from the list of students described in section 3, return to the page for Adding Camps.
Re-upload the .CSV file for enrollment and click "Update" to update the camp.
Make sure to check the list of students under the student view again, and repeat the process if necessary until the list of students is complete.
\end{document}
