\documentclass[letterpaper,10pt,serif,draftclsnofoot,onecolumn,compsoc,titlepage]{IEEEtran}

\usepackage{graphicx}                                        
\usepackage{amssymb}                                         
\usepackage{amsmath}                                         
\usepackage{amsthm}                                          
\usepackage{cite}
\usepackage{alltt}                                           
\usepackage{float}
\usepackage{color}
\usepackage{url}
\usepackage{subfiles}
\usepackage{blindtext}
%\usepackage[subpreambles=true]{standalone}
%\usepackage{import}
\usepackage{balance}
\usepackage[TABBOTCAP, tight]{subfigure}
\usepackage{enumitem}

\usepackage{geometry}
\geometry{margin=.75in}
\usepackage{hyperref}
\usepackage{tikz}
\usetikzlibrary{shapes, positioning, calc}
%\usetikzlibrary{shapes, positioning, calc}
\usepackage{caption}
\usepackage{listings}
%\usepackage[utf8]{inputenc}
%pull in the necessary preamble matter for pygments output

%% The following metadata will show up in the PDF properties
% \hypersetup{
%   colorlinks = false,
%   urlcolor = black,
%   pdfauthor = {\name},
%   pdfkeywords = {cs311 ``operating systems'' files filesystem I/O},
%   pdftitle = {CS 311 Project 1: UNIX File I/O},
%   pdfsubject = {CS 311 Project 1},
%   pdfpagemode = UseNone
% }
%credit to %http://tex.stackexchange.com/questions/48152/dynamic-signature-date-line
\newcommand*{\SignatureAndDate}[1]{%
    \par\noindent\makebox[2.5in]{\hrulefill} \hfill\makebox[2.0in]{\hrulefill}%
    \par\noindent\makebox[2.5in][l]{#1}      \hfill\makebox[2.0in][l]{Date}%
}%

\parindent = 0.0 in
\parskip = 0.1 in
\title{Final Progress Report}
\author{Shannon Ernst, Kyle Nichols, Javier Franco\\ Group 48 \\ 13 June 2017 \\ CS 463 Spring 2017}
\begin{document}
\maketitle
\begin{abstract}

\end{abstract}
\newpage
\tableofcontents
\newpage
\section{Introduction}
\section{Requirements Document}
\subsection{Original Document}
%\subfile{sections/design_doc}
\subsection{Changes to the Document}
\subsection{Gantt Chart Progression}
\section{Design Document}
\subsection{Original Document}
\subsection{Changes to the Document}
\section{Technology Review Document}
\subsection{Original Document}
\subsection{Changes to the Document}
\section{Weekly Blogs Over the Year}
\subfile{sections/fall_weekly_blogs}
\subfile{sections/winter_weekly_blogs}
\subfile{sections/spring_weekly_blogs}
\section{Poster}
\section{Documentation}
\subsection{Overall}
\subsection{Survey Generation}
\subsection{Database}
\subsection{Report Generation}
\section{Reflection}
\subsection{How to Learn}
For learning how to use new technology we did not use any books and we did not get outside help from other people that were not involved in our group. We learned new technology by researching the things that we needed on the fly. The majority of the things we needed were found in StackOverflow pages. Another website was useful was w3schools for looking up how to use a CSS style sheet, HTML, and PHP. This website was useful by providing code examples and also by generating the results of the code. below is the order of usefulness of the websites:
1.StackOverflow
2.w3schools
\subsection{What was Learned}
\subsection{Shannon Ernst: Survey Generation}
\subsection{Kyle Nichols: Database Management}
\subsection{Javier Franco: Report Generation}
The 

%\bibliographystyle{ieeetr}
%\bibliography{writing1}
\end{document}