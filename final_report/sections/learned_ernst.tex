\documentclass[../final.tex]{subfiles}

\begin{document}
\subsection{Shannon Ernst: Survey Generation}
Over the course of the year I have learned a lot, mostly 
nontechnical skills. This project was a web based app and 
was very similar to work I have done in the past. The most 
challenging aspect was learning how to dynamically generate 
HTML from scratch. If I were to do this aspect again I would 
put more time and energy into setting up a javascript framework 
that would have handled this for me. I have worked in AngularJS 
quite a bit for my interships but I had never set it up from 
the ground up and for the purposes of time on this project I did 
not push myself to figure out how to set it up. I should have looked 
more in depth to other frameworks and possibly talked to Todd Schechter,
 the IT Director, about how to properly set the frameworks up on 
 the engineering server. After figuring out how to generate HTML 
from scratch the next thing I learned was how to use JSON objects 
through out a web app for faster data manipulation. Everything for 
survey generation and survey taking is based off of stored JSON objects.
The storage of these objects was slightly complicated as we had to make 
them work with PHP 5. If I were to do this project again on my own I would 
endavor to use more current languages for the back end such as Node.js. 
Unfortunately, this project was restricted to where it could be hosted 
due to it having minor information as well as the amount of money the 
client was willing to spend. \\\\
A web application was probably not the best solution to the issue 
presented by our clients. They wanted a database to crunch their survey 
data faster. What we should have done was instruct them on how to use 
Qualtrics or another preexisting survey software and came up with a 
solution to what they could to tabulate their paper surveys faster. This 
could have been some form of OCR or scantron for paper surveys that would 
automatically tabulate the results and store them digitally. This probably 
would have been less work for our client in the long run and would have 
been a more interesting project for us instead of reinventing something that 
has multiple known, effective solutions. \\\\
More than anything I learned how to work with people who were not necessarily 
competent through out this project. I have been spoiled in having internships 
with passionate people who are good at what they do. My team for capstone was 
obedient but did not always know how to think through a problem. This was often 
frustrating when things were tight on time. For the most part, the team was managed 
 well and we kept to a consistant schedule of working during the registered class time.
I learned to always be catious with stating how much work can be done in a given time 
because inevitably this will often not happen. I learned that frequent communication 
with the client is the best way to spot issues early. I also learned that it is better 
to ask for help earlier rather than later. Ultimately, capstone was an exercise in 
making do with what you have to the best of your ability.  
\end{document}
