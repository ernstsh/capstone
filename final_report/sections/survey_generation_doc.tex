\documentclass[../final.tex]{subfiles}

\begin{document}
\subsubsection{Overview}
Survey generation is the first step in using the website. Without a created 
survey there is no way to collect data and analyze it. On a clean implementation 
of the database, there will not be any surveys to preload. To create a survey the 
admin will navigate from the dashboard to the Edit/Create Survey page. On this page 
the admin will provide the appropriate title for the survey via text box, associated 
camp via drop down, and indicate the type of survey via radio buttons. Once this 
intial data has been provided, the admin can add questions by selecting from the 
Add Question drop down and may select either text, matrix or multiple choice. This 
question will be added to the survey and the admin will be able to fill in the 
appropriate information pertaining to that question. Once the admin has finished adding 
questions they can preview the survey by clicking Preview. This will open a pop up 
window with the html that will be generated for a user taking the survey. Once the admin 
is satisfied with the survey they can save the survey via the Save button. Saving the 
survey does not redirect the user from the page. To exit the user will have to click 
exit. \\
To recall a survey to be edited, the admin can select from the drop down at the top 
of the Edit/Create page the survey they want to edit. This will reload all information 
associated with this survey into the appropriate question templates to allow for 
editing. The admin will have to save the survey after editing to ensure that the 
information is updated. The admin also has the ability to mark questions as frequently 
used. If a question is marked frequently used it will appear in a frequently used 
drop down menu to be seleceted to add to the survey. These questions are also editable 
within the survey but can never be updated in the database in any way. \\ 
\subsubsection{Data Flow}
The following image illustrates the flow of data for creating a survey.
There are two entry points to for ultimately saving a survey: editing 
a survey and creating from scratch. The layers for programming languages 
is HTML, javascript and PHP/mySQL. \\
\begin{tikzpicture}[node distance=2cm]

\node (start) [startstop] {Start};
\node (pro1) [process, below of=start] {Admin requests to edit a survey by selecting from the drop down};
\node (pro2) [process, below of=pro1] {Database returns survey JSON};
\node (pro3) [process, below of=pro2] {Admin provides information for creating survey};
\node (pro4) [process, below of=pro3] {JSON object created by scraping web page};
\node (pro5) [process, below of=pro4] {Add survey to database};
\node (stop) [startstop, below of=pro5] {Stop};
\draw [arrow] (start) -- (pro1);
\draw [arrow] (pro1) -- node[anchor=east] {POST} (pro2);
\draw [arrow] (pro2) -- node[anchor=east] {Populates webpage with survey information} (pro3);
\draw [arrow] (pro3) -- node[anchor=east] {Submit} (pro4);
\draw [arrow] (pro4) -- node[anchor=east] {POST} (pro5);
\draw [arrow] (pro5) -- (stop);

\end{tikzpicture}
\\
The following image illustrates the flow of data during survey taking. \\
\begin{tikzpicture}[node distance=2cm]

\node (start) [startstop] {Start};
\node (pro1) [process, below of=start] {Student selects a camp};
\node (pro2) [process, below of=pro1] {Student selects name};
\node (pro3) [process, below of=pro2] {Student takes survey};
\node (pro4) [process, below of=pro3] {JSON object created by scraping web page};
\node (pro5) [process, below of=pro4] {Add response to database};
\node (stop) [startstop, below of=pro5] {Stop};
\draw [arrow] (start) -- (pro1);
\draw [arrow] (pro1) -- node[anchor=east] {POST for students in camp} (pro2);
\draw [arrow] (pro2) -- node[anchor=east] {POST to generate survey} (pro3);
\draw [arrow] (pro3) -- node[anchor=east] {Submit} (pro4);
\draw [arrow] (pro4) -- node[anchor=east] {POST} (pro5);
\draw [arrow] (pro5) -- (stop);

\end{tikzpicture}
\end{document}