\documentclass[../final.tex]{subfiles}
\begin{document}
\section{Essential Code Listings}
\subsection{Survey Generation}
\subsection{Database}
The following sections show code examples that highlight the database development in the project.
\subsubsection{Creating Tables}
Below is a \texttt{CREATE TABLE} MySQL query.
There are several tables in our database, and they all were created through a \texttt{CREATE TABLE}, though some were edited later.
This table, Camp, is shown here particularly for highlighting several elements of table creation in our system.
\begin{lstlisting}
CREATE TABLE `Camp` (
  `camp_id` int(11) NOT NULL,
  `title` varchar(255) NOT NULL,
  `start_date` date NOT NULL,
  `end_date` date NOT NULL,
  `pre` int(11) DEFAULT NULL,
  `post` int(11) DEFAULT NULL,
  `enrollment` varchar(4096) DEFAULT NULL,
  PRIMARY KEY (`camp_id`),
  KEY `pre` (`pre`),
  KEY `post` (`post`)
) 
\end{lstlisting}
The ID is generated as an integer that cannot be empty, since IDs are essential to reference each table entry uniquely.
This unique identification is why camp\_id is created as a \texttt{PRIMARY KEY}.
Since camps have a variety of information, they are all listed here with their appropriate types (int, varchar, date, etc.).
Some of these attributes are important to the functionality of the system, which is why they are set as \texttt{NOT NULL}: they cannot be created empty.
\emph{pre} and \emph{post} are foreign keys since they are linked to survey IDs in the Survey table.
\subsubsection{Interacting with the Database Through PHP}
Below is the \emph{save\_db.php} file, which is a great example of the structure of our database interaction.
The database information has been removed for privacy purposes, as well as some other minor comments and minor code.
\begin{lstlisting}
<?php
error_reporting(-1);

$obj = stripslashes($_POST['x']);
echo $obj;
$ar = json_decode($obj);
$conn = new mysqli("", "", "", "");

# ADD SURVEY
$sql = "INSERT INTO Survey(survey_id, title, arr_questions, survey_type) VALUES (?,?,?,?)";
if($statement = $conn->prepare($sql)){
	do {
           	$survey_id = rand(1000, 5000);
           	$result = $conn->query("SELECT * FROM Survey WHERE survey_id='".$survey_id."'");
           } while (!$result);
	$title = $ar->title;
	$type = $ar->type;

	// Generate the array of questions string
	$arr_questions = json_encode($ar->questions);

	$statement->bind_param('isss', $survey_id, $title, $arr_questions, $type);
	$statement->execute();
	$statement->close();
}
else {
	printf("Error: %s\n", $conn->error);
}

# ADD PRE/POST TO CAMP
if ($ar->type == "pre") {
   	$sql = "UPDATE Camp SET pre='".$survey_id."' WHERE Camp.camp_id='".$ar->camp."'";	
	$result = $conn->query($sql);

} else if ($ar->type == "post") {
   	$sql = "UPDATE Camp SET post='".$survey_id."' WHERE Camp.camp_id='".$ar->camp."'";
   	$result = $conn->query($sql);
} else {
  	 echo "Error in pre/post type in JSON. Type: ".$ar->type."\n";
}

if ($ar->type == "pre" || $ar->type == "post") {
  	 if ($result) {
      		echo "Successfully updated Camp row.\n";
  	 } else {
      		echo "Error: ".$conn->error." <br>";
  	 }
}

$conn->close();
?>
\end{lstlisting}
The database is accessed using \emph{mysqli} objects.
We generate MySQL queries through strings, sometimes preparing them with empty values to be filled in later.
INSERT, SELECT, and UPDATE queries are all done here and elsewhere, often with information that is passed in through a POST method.
In this case, the POST is a JSON object represented as a string, which we decode into a usable object to get data through the rest of the code.
When everything is done, we close the database connection.
Often, text echoed in the PHP code is sent back to JavaScript to be handled as response text from the XMLHttpRequest that called the POST method.
\subsection{Report Generation}
\end{document}
