\documentclass[letterpaper,10pt,serif, draftclsnofoot,onecolumn, compsoc, titlepage]{IEEEtran}

\usepackage{graphicx}
\usepackage{amssymb}
\usepackage{amsmath}
\usepackage{amsthm}

\usepackage{alltt}
\usepackage{float}
\usepackage{color}
\usepackage{url}

\usepackage{balance}
\usepackage[TABBOTCAP, tight]{subfigure}
\usepackage{enumitem}
\usepackage{pstricks, pst-node}

\usepackage{geometry}
\geometry{margin=.75in}

\usepackage{hyperref}

\begin{document}

% Adding these from the template in Annex C of the standard
% Some may not apply, but I'm sure most will
\section{Frontspiece}
\subsection{Date of issue and status}
\subsection{Issuing organization}
\subsection{Authorship}

\subsection{Change history}

\section{Introduction}
\subsection{Purpose}
\subsection{Scope}
\subsection{Context}
\subsection{Summary}

\section{References}

\section{Glossary}

\section{Body}
\subsection{Identified stakeholders and design concerns} 


%Report Generation 
\subsection{Design viewpoint \emph{1}}
The context viewpoint will be used to govern the viewpoint for generating a report. 
The report webpage will allow administrator users which also include the stakeholders to query information from the database to generate a report.
The queries that the users will be able to generate will be based on camp, club, program, student, demographic information, and survey questions fields. 
To generate a report a user will enter the information that they want to query manually into the input fields in a form and they will be able to add additional input fields to the form to query additional information.
The user will be able to submit their query by clicking a submit button which will return the results of the query. 
In addition, users will have the option to choose whether or not they want a summary of the queried results which includes the average, count, sum, etc. 

\subsection{Design view \emph{1}}
To create the form for the report generation page AngularJS, HTML, and PHP will be used. 
AngularJS requires HTML in order to define the application's user interface \cite{Lau}. 
The reason for using AngularJS is that it can be used to create a dynamic webpage instead of having a static webpage.
A disadvantage of a static webpage is that in order to update a webpage the HTML content must be changed manually \cite{Webpage}. 
This means that if a user wanted to make additional queries for a certain field such as suvery question the developer would have to add the input fields manually which is a terrible way for creating the search form.
AngularJS has the ability to create dynamic addition field/fieldset control for adding and removing additional fields without relying on a programmer to do this \cite{AddField}. 
The HTML webpage will require input fields to query information based on camp, club, program, student, survey question, demographic information, and student responses. 
Each of these input fields will have a button that will create addiitonal input fields of the selected input field. 
Each input field will have their own individual button that will delete the input field incase the administrative user no longer needs the input field for querying additional information. 
Both buttons will have an ng-click AngularJS directive which will call their corresponding function for deleting or adding a new input field. A controller will be used to modify the HTML expressions and the \$scope service will be used to detect any changes\cite{Rav}. 
PHP will be used to query the database using the information that the user entered in the input fields. 
This will be done using the SELECT statement in PHP to query the information from the database based on the input fields. If the any input fields are left empty they will be ignored when the query is submitted via a button. 

%Saving a report  
\subsection{Design viewpoint \emph{2}}
 The design view that will be used to save a report generated by an administrative user will be a conte
\subsection{Design view \emph{2}}
For saving a report generated by an administrative user PHP will be used to save and retrieve the query results for the report.
Before saving and retrieving query results a table must be established in the database for storing only query results for the reports. 
Saving the results into a table will store the results permanently \cite{Microsoft}. 
To store information into a certain database table PHP uses the INSERT INTO statement. In addition, for retrieving information from a database table PHP uses the SELECT statement to select data from one or more tables. 
The INSERT INTO statement will be used to stor
%Editing
\subsection{Design viewpoint \emph{3}}
The viepoint that will be used to govern the editing of a report will be 
\subsection{Design view \emph{3}}


%Printing 
\subsection {Design viewpoint \emph{4}}
The design viewpoint that will be used for printing a report will be  the context viewpoint. 
The user interface for generating reports will allow users to print out a report that they create or have saved.
The user will be able to print their report by clicking on a print icon near their report that is visible.
When the icon is clicked it will display a print dialogue box that will allow the user to edit the print settings and it will display the report that will be printed.
The user will have the option to print on both sides of a page, enter a page range, change the margins, change orientation, and etc. 
When a report is printed it should only print the report by excluding unrelated content on the webpage that is not part of the report. 
\subsection{Design view \emph{4}}
The tool that will be used to address the design concerns mentioned previously in the context viewpoint will be Javascript and a CSS style sheet. 
By only using Javascript to printout a report it will result in printing out the entire webpage which is not the desired result and it is the reason why a CSS style sheet is necessary.
CSS is used to create two separate style sheets one for the webpage and another for printing by leaving out unwanted content from the webpage \cite{Javascript}. 
To print a report the code href="javascript:window.print()" is added to a button or a link tag \cite{Javascript}. 
In our case we will be using a button tag instead of a link since it is more visually pleasing compared to using a hyperlink. 
When the button is clicked it will display the print dialogue box which will allow a user to adjust the print settings to suite their needs. 

\subsection{Design rationale}

\begin{thebibliography}{2}
\bibitem{Javascript}"How to Print a Specific Part of a HTML Page Using CSS." \em{ARCLAB}. N.p., n.d. Web. 11 Nov. 2016. 
\bibitem{Webpage} "Web Design Choices." \em{SpiderWriting}. Web. 25 Nov. 2016. 
\bibitem{AddField}  Muhammed, Shanid. "AngularJS | Adding Form Fields Dynamically". \em{shanidkv.com}. N.O, Web. 26 Nov. 2016. 
\bibitem {Lau} Lau, Dmitri. "10 Reasons Why You Should Use AngularJS." \em{sitepoint}. N.p., 5 Sept. 2013. Web. 26 Nov. 2016.
\bibitem{Rav} C, Ravindra T. "Connecting to Database Using AngularJS." \em{Code Project}. N.p., 25 Mar. 2016. Web. 26 Nov. 2016.

\bibitem{Microsoft} "Storing Query Results in a Table, Array, or Cursor." \em{Microsoft}. N.p., n.d. Web. 27 Nov. 2016.

\end{thebibliography}

\end{document}